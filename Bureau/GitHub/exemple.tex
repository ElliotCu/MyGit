\documentclass{article}

\usepackage[T1]{fontenc} 
\usepackage[utf8]{inputenc}
\usepackage[francais]{babel}


\title{First LateX Document}
\author{WARDEH Amir}

\begin{document}

\maketitle

\begin{abstract}
Ceci est le résumé de l'article
\end{abstract}

\section{Première section}
\subsection{Sous-section 1}
Ceci est mon \textbf{premier} document en \LaTeX,
écrit le \today{} en TP d’environnement de programmation,
avec le mode « article ».
\subsubsection{sous-sous section}
Changement de la taille et de type:

{\textbf{\emph{\large Ceci est mon \textbf{premier} document en \LaTeX,
écrit le \today{} en TP d’environnement de programmation,
avec le mode « article ».}}}
\subsubsection {sous-sous section}
\subsubsection{sous-sous section}
\subsubsection{sous-sous section}

\begin{itemize}
\item item 1
\item item 2
\item item 3
\end{itemize}


\subsection{Sous-section 2}
\section{Deuxième section}
\subsubsection{sous-sous section}
\subsubsection{sous-sous section}
\subsubsection{sous-sous section}

\begin{description}
\item [new item 1] different item 1
\item [new item 2] same item
\item [new item 3] latex item
\end{description}
\subsubsection{sous-sous section}

\begin{enumerate}
\item myitem 1
\item myitem 2
\item myitem 3
\end{enumerate}
\section{Troisième section}

\section{Résultats}
\label{resultats}
Les résultats sont préseentés dans la section~\ref{resultats}
Le numéro de la page : \pageref{resultat}
\tableofcontents
\begin{thebibliography}{1}

\bibitem{diday-1982}
E.~Diday, J.~Lemaire, J.~Pouget, and F.~Testu.
\newblock {\em \'El\'ements d'analyse de donn\'ees}.
\newblock Dunod, 1982.

\bibitem{heckbert-1982}
P.~Heckbert.
\newblock Color image quantization for frame buffer display.
\newblock In {\em Proceedings of SIGGRAPH'82}, pages 297--307, 1982.

\bibitem{preparata-1988}
F.~Preparata and M.~I. Shamos.
\newblock {\em Computational geometry, an introduction}.
\newblock Springer-Verlag, 1988.

\bibitem{tutte-1963}
W.T. Tutte.
\newblock A census of planar maps.
\newblock {\em Canad.J.Math.}, 15:249--271, 1963.

\end{thebibliography}

Commande:
\newcommand{\paire}[2]{(#1,#2)}
\paire{1}{2}, \paire{2}{3}

items spéciaux:
{\renewcommand{\labelitemi}{$\heartsuit$}
\begin{itemize}
\item joli item1
\item joli item2
\end{itemize}

\newcounter{compteur}
\setcounter{compteur}{0}
première valeur: \thecompteur\\
\addtocounter{compteur}{1}
deuxième valeur: \thecompteur\\

\newenvironment{question}{\noindent{\Large \textbf{Question --}}}{Fin\\[.2cm]}
\begin{question}
première question
\end{question}
\begin{question}
deuxième question
\end{question}

\section{Texte Mathématique}
\Large{
$
z_{1} = \sqrt{$x_{1}^{2}$+$y_{1}^{2}$}
$

    
$T(I) = \sum_{i=0}^{I} \frac{h(i)}{M}$
    

$X = \frac {\int_{\lambda min}^{\lambda max} E(\lambda) S(\lambda) \overline{x}(\lambda)d\lambda}{\int_{\lambda min}^{\lambda max}S(\lambda)\overline{y}(\lambda)d\lambda}$
}

\end{document}

